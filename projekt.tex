\documentclass[12pt]{article}
\usepackage[slovak]{babel}
\usepackage[utf8x]{inputenc}
\usepackage[T1]{fontenc}
\usepackage{hyperref}

\title{Predikovanie gramatických kategórii slov na základe word2vec}
\author{
        Mária Vajdová
}
\date{\today}


\begin{document}
\maketitle

\begin{abstract}

Projekt sa zaoberá zisťovaním, či existuje vzťah medzi slovnými vektormi vytvorenými knižnicou \texttt{word2vec} 
a morfologickými kategóriami slov v slovenskom jazyku (slovný druh, rod, číslo a pod.) Práca sa skladala zo štyroch častí - 
natrénovanie slovných vektorov na základe slovenskej wikipédie, získania morfologických anotácii slov z Jazykovedného ústavu Ľudovíta Štúra, následne vytvorenia trénovacích a testovacích dát z týchto zdrojov a natrénovania predikcie gramatických kategórii na týchto dátach.

Výsledkom práce je porovnanie niekoľkých metód predikcie na základe word2vec. Zaujímavým výsledkom je nezávislosť veľkosti slovných vektorov a kvality predikcie a napr. dobrá schopnosť rozoznávať slovné druhy u predikovaných slov a oveľa horšia schopnosť rozoznávať iné gram. kategórie. 

\end{abstract}

\section{Trénovanie slovných vektorov pomocou word2vec}
\subsection{Získavanie dát}
Ako text pre trénovanie slovných vektorov som použila dump slovenskej wikipédie dostupný v tom čase na \url{https://dumps.wikimedia.org/backup-index.html}. Predspracovala som ho použitím upraveného perlového skriptu používaného na získavanie plain-textu pre word2vec z wikipédie na \url{http://mattmahoney.net/dc/textdata.html}. Môj upravený skript je dostupný v repozitári k projektu - oproti pôvodnému skriptu zachováva diakritiku a prekladá čísla do slovenčiny(\texttt{word2vec} lepšie pracuje so slovnými reprezentáciami čísel, napr. 12 - \texttt{jedna dva}). 

\subsection{Trénovanie slovných vektorov}
Na trénovanie slovných vektorov som použila štandartnú implementáciu knižnice \texttt{word2vec} dostupnú na \url{https://code.google.com/archive/p/word2vec/}. Knižnica slúži na preklad slov prirodzeného jazyka do niekoľkorozmerných vektorov reálnych čisel reprezentujúcich význam slov. Predikcia pre jednotlivé slová sa vykonáva na základe kontextu, v akom sa vyskytujú v texte. Predpokladá sa, že slová s podobným významom majú podobné slovné vektory. Morfologické vzťahy medzi týmito slovami zatiaľ veľmi podrobne skúmané neboli. 

\subsection{Výstup word2vec}
Celkový počet slov braných do úvahy v danom texte slovenskej wikipédie je $52887797$, čo predstavuje $187659$ unikátnych slov. Trénovala som slová s nasledovnými parametrami, pričom som trénovala vektory rôznej dĺžky kvôli následnému zisteniu korelácie s morfologickými znakmi slov. 
Dané parametre, s ktorými som word2vec trénovala sú uvedené v tabuľke \ref{tab:v}. Natrénované slovné vektory nie sú kvôli svojej veľkosti súčasťou repozitára. Pre ďalšie spracovanie som používala iba slovné vektory, ktorých počet výskytov slov v slovenskej wikipédii bol aspoň 100, čo predstavuje 33401 slov. To je z dôvodu, že pre ďalšie použitie v diplomovej práci, ktorá by mala nadväzovať na tento projekt sa predpokladá použitie rozsiahlejšej databázy slov. 

\begin{table}
\begin{tabular}{|l|l|l|l|}
\hline
Dĺžka vektorov & Počet iterácii alg. & Použitý alg. & Min. počet výskytov slova \\ \hline
50 & 10 & skip-gram & 10 \\ 
100 & 10 & skip-gram & 10 \\ 
200 & 10 & skip-gram & 10 \\ 
500 & 5 & skip-gram & 10 \\
\hline
\end{tabular}
\caption{Parametre natrénovaných slovných vektorov}\label{tab:v}
\end{table}

\section{Morfologické anotácie slovenských slov}

Táto časť práce mi osobne zabrala najviac času, najmä kvôli neprehľadnosti stránky \url{http://korpus.juls.savba.sk/}, kde sa morfologické anotácie slov nachádzajú. Stránka poskytuje webové rozhranie, ktorým sa dajú vyhľadávať tvary slov slovenského jazyka a rovnako poskytuje databázu, kde sa dá pristupovať k rôznym korpusom. Až po konzultácii so zamestnancami som sa dostala k dátam, ktoré som reálne potrebovala a to \url{http://korpus.juls.savba.sk/morphology_database.html}, kde sú dostupné morfologické anotácie asi milióna slov slovenského jazyka. Ja som používala voľne prístupnú morfologickú databázu, verziu 2015-02-05. 

Z danej databázy som vybrala slová, pre ktoré máme aj natrénované slovné vektory (tie, ktoré mali výskyt aspoň 100 vo wikipédii) a prečíslovala morfologické anotácie pre slovný druh, rod, číslo, pád a osobu (pri slovesách). Skript, ktorý to robil je k dispozícii v repozitári.  

Problémom s týmito dátami, ktoré bude pravdepodobne v budúcnosti treba lepšie riešiť je viacznačná morfologická anotácia pre jedno slovo (jeden tvar slova môže zodpovedať viacerým pádom slova a pod.). Pri spracovaní vstupu som vyhadzovala všetky duplicitné dvojice slovo - morfologická anotácia (daný dokument neobsahuje duplicity, my sme ale nebrali do úvahy všetky morfologické kategórie, z toho dôvodu duplicity vznikli). 

Vzniknuté dáta môžem poskytnúť, ale nie sú súčasťou repozitára, kvôli tomu, že nemám na ich zverejňovanie práva. 

\section{Predikovanie morfologických kategórii na základe slovných vektorov}
\subsection{Príprava vstupu}
Vstupné slová som si náhodne rozdelila na trénovaciu množinu veľkosti $5000$ a zvyšok ako testovaciu množinu. (Skúšala som aj nenáhodné rozdelenie, ktoré zobralo prvých $5000$ slovných tvarov na trénovanie, ale jeho výsledky neboli lepšie). Trénovanie som skúšala spúšťať na vektoroch rôznych veľkostí(50-500), ale trénovanie dlhších vektorov dosahovalo iba mierne lepšie výsledky. 

\subsection{Spôsoby trénovania}
Za najvhodnejší trénovací algoritmus pre tento problém som považovala SVM, konkr. Support Vector Classification s nastaveným 'one-vs-one' tvarom rozhodovacej funkcie. Zároveň som zapla podporu pre pravdepodobnosti jednotlivých tried. Pre SVM som sa rozhodla z dôvodu, že samotné slovné vektory podľa mňa najlepšie zodpovedajú spôsobu, akým pracuje SVM, teda rozdeľuje priestor vektorov na podpriestory. Rovnako som skúšala trénovať rozhodovacie stromy, ktoré síce dokázali lepšie nafitovať vstup, ale testovacie skóre bolo oveľa horšie. 

Trénovala som niekoľko kategórii dát:
\begin{enumerate}
\item slovný druh slova
\item rod podstatného mena
\item rod prídavného mena
\item číslo podstatného mena
\item číslo prídavného mena
\item pád podstatného mena
\item pád prídavneho mena
\item osobu slovesa\
\end{enumerate}
Uvádzam výsledky pre nastavenie $5000$ trénovacích slov, $71301$ testovacích slov, trénovanie na vektoroch dĺžky 500. DO úvahy som pri každom trénovaní aj testovaní brala iba slová, ktoré obsahovali danú gram. kategóriu.

\subsection{Vyhodnocovanie}

Pre každú meranú kategóriu som si zrátala trénovacie skóre, testovacie skóre a multiclass skóre, ktorého výpočet je popísany nižšie. 


\begin{table}[]
\centering
\caption{My caption}
\label{my-label}
\begin{tabular}{lllll}
\textbf{N} & \textbf{Trénovacie skóre} & \textbf{Testovacie skóre} & \textbf{Multiclass skóre} & \textbf{} \\
1          & 0.785951589938            & 0.742627459276            & 0.172624022994            &           \\
2          & 0.691494981511            & 0.666489965922            & 0.0922757117145           &           \\
3          & 0.546652609383            & 0.490822868183            & 0.125297233031            &         \\
4          & 0.808785529716            & 0.781045323553            & 0.100273785079            &           \\
5          & 0.821489482661            & 0.773028452644            & 0.103031776913          &           \\
6          & 0.242424242424            & 0.233968499508            & 0.0581753455812            &         \\ 
7          & 0.282407407407            & 0.253817185895            & 0.0214721913409          &           \\
8          & 0.520661157025            & 0.424694708277            & 0.26275971093            &         \\ 
\end{tabular}
\end{table}


\subsubsection{Multiclass score}
Multiclass skóre som vytvorila tak, že pre každý vstup som urobila zoznam všetkých možných správnych výstupov a následne som sa pozrela na pravdepodobnosti, ktoré pre daný vstup vytvorilo SVM. Skóre pre daný vstup som potom počítala ako:
\[\frac{|dobre|-0.5*|zle|}{|triedy|},\]
kde $|dobre|$ označuje počet správne určených výstupov aspoň s pravdepodobnosťou $\frac{1}{|triedy|}$, kde $|triedy|$ označujú celkový počet klasifikačných tried a zlé označujú počet nesprávne určených výstupov, ktoré majú tiež aspoň pravdepodobnosť $\frac{1}{|triedy|}$. Skóre som pre celú množinu spriemerovala. 


\subsubsection{Multilabel vyhodnocovanie}
Skúšala som trénovať aj multilabel vyhodnocovanie pomocou OneVsRestClassifier, ale jeho výsledky tiež neboli veľmi dobré. 

\subsection{Výsledky}
Celkovo trénovanie nedopadlo až tak dobre, skúšala som viaceré metódy, z toho SVM mala najlepšie testovacie výsledky. Z výsledkov vyplýva, že vo vektoroch sa odzrkadľuje slovný druh slova, rovnako aj rod a číslo podstatného a prídávneho mena, ale je problém s určovaním pádu. Osoba slovies sa tiež veľmi nepodarila určiť. 

\section{Úspešnosť projektu}
Projekt považujem skôr za neúspešný, bol pre mňa veľký problém spracovávať viacznačné dáta - slová, ktoré reprezetnovali viacero gramatických kategórii naraz. V budúcnosti by bolo dobré lepšie sa zamyslieť nad predprípravou dát na ich spracovanie, príp. úpravou word2vec algoritmu. 

\end{document}
  